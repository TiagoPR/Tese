\chapter*{Resumo}

% O que se quer no resumo então?

%Escrever aqui o resumo (pt)
A gestão do consentimento de acesso e processamento de dados é uma das questões centrais para garantir a privacidade de todos os utilizadores, tendo em consideração regulamentação já existentes para o cumprimento da mesma. Esta dissertação explora \acrfull{cmp} existentes, destacando as suas características, limitações e potencialidades de forma a criar soluções mais auditáveis e transparentes. 

Foi realizada uma análise ao tratamento de dados, incluindo plataformas como Osano, Cookiebot, Klaro.js, entre outras, englobando soluções avançadas, como o caso do Google Consent Mode, e abordagens complementares para a gestão deste mesmo consentimento, baseadas em tecnologias como \textit{blockchain} e contratos inteligentes.
Verificou-se que as soluções já existentes, embora robustas, apresentam limitações no que toca à auditoria e transparência do fluxo de dados entre clientes e provedores de serviço. 
Assim, foi proposta uma solução que permite recolher consentimentos verificáveis, isto é, que servem como prova para os utilizadores no caso de não conformidade com as suas escolhas. 

Os resultados obtidos permitem demonstrar uma prova de conceito de um sistema que integra a recolha de consentimento com mecanismos de prova, permitindo garantir que as autorizações recolhidas são armazenadas de forma verificável, imutável e não repudiável. 
Desta forma, esta dissertação propõe o desenvolvimento de uma solução que reforça a confiança dos utilizadores no tratamento dos seus dados, promovendo uma maior transparência e autonomia na gestão destes mesmos dados por parte do utilizador.

\paragraph{Palavras-chave} consentimento, \acrshort{cmp}, \acrshort{rgpd}, auditoria, prova, transparência, auditabilidade, imutabilidade, privacidade, \textit{open source}, assinaturas digitais, certificados, JWS, não-repúdio

% Demasiado genericos, rever

\cleardoublepage

\chapter*{Abstract}

%Write abstract here (en)
The management of consent for data access and processing is one of the central issues in ensuring user privacy, taking into account existing regulations that govern compliance. This dissertation explores existing \acrfull{cmp}s, highlighting their characteristics, limitations, and potential to create more auditable and transparent solutions. 

An analysis was carried out on data processing, including platforms such as Osano, Cookiebot, and Klaro.js, among others, encompassing advanced solutions such as Google Consent Mode, as well as complementary approaches to consent management based on technologies like \textit{blockchain} and smart contracts. It was found that existing solutions, although robust, present limitations in terms of the auditability and transparency of data flows between clients and service providers. Thus, a solution was proposed that enables the collection of verifiable consents, that is, consents that serve as proof for users in cases of non-compliance with their choices. 

The results obtained demonstrate a proof of concept of a system that integrates consent collection with verification mechanisms, ensuring that the collected authorizations are stored in a verifiable, immutable, and non-repudiable way. In this way, this dissertation proposes the development of a solution that strengthens user trust in data processing, promoting greater transparency and autonomy in the management of their personal data.

\paragraph{Keywords} consent, \acrshort{cmp}, \acrshort{gdpr}, auditing, transparency, auditability, proof, immutability, privacy, \textit{open source}, digital signatures, certificates, JWS, non-repudiation

\cleardoublepage
