\chapter{Planeamento}

\begin{table}[H]
\centering
\renewcommand{\arraystretch}{1.3} % Ajusta a altura das linhas
\setlength{\tabcolsep}{4pt} % Ajusta o espaçamento entre as colunas
\resizebox{\textwidth}{!}{%
\begin{tabular}{| c | c | c | c | c | c | c | c | c | c | c | c | c |}
\hline
\textbf{Tarefa} & \textbf{Out} & \textbf{Nov} & \textbf{Dez} & \textbf{Jan} & \textbf{Fev} & \textbf{Mar} & \textbf{Abr} & \textbf{Mai} & \textbf{Jun} & \textbf{Jul} & \textbf{Ag} & \textbf{Set} \\
\hline
\textit{Background} e \acrshort{eda} & $\bullet$ & $\bullet$ & $\bullet$ & & & & & & & & & \\
\hline
Estudo de abordagens & & $\bullet$ & $\bullet$ & $\bullet$ & & & & & & & & \\
\hline
Análise de Fluxos de Dados & & & & $\bullet$ & $\bullet$ & $\bullet$ & & & & & & \\
\hline
Desenho da Solução & & & & & & $\bullet$ & $\bullet$ & & & & & \\
\hline
Desenvolvimento da Solução & & & & & & & $\bullet$ & $\bullet$ & $\bullet$ & $\bullet$ & & \\
\hline
Otimização do sistema & & & & & & & & & & $\bullet$ & $\bullet$ & \\
\hline
Redação e Revisão da Dissertação & & & & & & & & & & & $\bullet$ & $\bullet$ \\
\hline
\end{tabular}%
}
\caption{Plano de atividades.}
\label{tab:plano-atividades}
\end{table}


\begin{enumerate} 
    \item \textbf{Background e \acrlong{eda} (Outubro - Dezembro 2024)} \\
    \textbf{Objetivo}: Realizar uma pesquisa aprofundada sobre as tecnologias e metodologias existentes no domínio de \acrfull{cmp}. A pesquisa centrada na análise das soluções atuais, como Osano, Cookiebot, Tarteaucitron.js e Google Consent Mode. Também será explorado o impacto das regulamentações como o \acrshort{rgpd} na gestão de consentimento.
    
    \textbf{Tarefas}:
    \begin{itemize}
        \item Estudo e documentação das práticas atuais de gestão de consentimento.
        \item Análise de regulamentações como o \acrshort{rgpd} e sua aplicação nas \acrshort{cmp}'s.
    \end{itemize}

    \item \textbf{Estudo de abordagens (Novembro 2024 - Janeiro 2025)} \\
    \textbf{Objetivo}: Examinar abordagens e técnicas existentes para a gestão de consentimento e auditoria de dados, considerando alternativas tecnológicas que possam ser incorporadas na solução proposta.
    
    \textbf{Tarefas}:
    \begin{itemize}
        \item Análise de técnicas de recolha e armazenamento de consentimento.
        \item Investigação sobre o uso do \textbf{CookieChimp}, incluindo testes e análise da sua \acrshort{api}.
        \item Leitura de artigos sobre \textbf{\textit{blockchain}} e \textbf{smart contracts} para possível uso na gestão de consentimento e auditoria.
    \end{itemize}

    \item \textbf{Análise de Fluxos de Dados (Janeiro - Março 2025)} \\
    \textbf{Objetivo}: Analisar os fluxos de dados típicos de um cliente até à plataforma de gestão de consentimento e avaliar as tecnologias disponíveis para garantir a auditabilidade desses fluxos, com especial ênfase no uso de \textit{blockchain} e contratos inteligentes.
    
    \textbf{Tarefas}:
    \begin{itemize}
        \item Estudo de casos práticos de fluxos de dados em plataformas \acrshort{cmp}.
        \item Seleção das tecnologias mais adequadas para garantir a transparência e a rastreabilidade desses fluxos.
        \item Análise dos desafios da auditoria de dados com \textit{blockchain} e a viabilidade do uso de smart contracts.
    \end{itemize}

    \item \textbf{Desenho da Solução (Fevereiro - Março 2025)} \\
    \textbf{Objetivo}: Desenvolver o design da solução auditável de gestão de consentimento. A solução deverá integrar a recolha de consentimento e mecanismos de auditoria, com a possibilidade de utilização de tecnologias como \textit{blockchain} para garantir a transparência.
    
    \textbf{Tarefas}:
    \begin{itemize}
        \item Desenho do modelo de dados e arquitetura do sistema.
        \item Definição das principais funcionalidades da plataforma.
    \end{itemize}

    \item \textbf{Desenvolvimento do Protótipo (Março - Maio 2025)} \\
    \textbf{Objetivo}: Desenvolver um protótipo funcional do sistema, com foco na recolha, armazenamento e auditoria do consentimento de dados.
     
    \textbf{Tarefas}:
    \begin{itemize}
        \item Implementação inicial do protótipo, com funcionalidades básicas de recolha de consentimento.
        \item Armazenamento de consentimento através da \textit{blockchain}.
        \item Integração com contratos inteligentes.
        \item Testes preliminares de usabilidade.
    \end{itemize}

    \item \textbf{Otimização do Sistema (Junho - Julho 2025)} \\
    \textbf{Objetivo}: Melhorar a usabilidade e desempenho do sistema para garantir que a plataforma seja robusta e eficiente para uso real.
     
    \textbf{Tarefas}:
    \begin{itemize}
        \item Refinamento do protótipo com base nos testes e \textit{\textit{feedback}s} obtidos.
        \item Melhorias na interface de utilizador e na arquitetura do sistema.
    \end{itemize}

    \item \textbf{Redação e Revisão da Dissertação (Junho - Setembro 2025)} \\
    \textbf{Objetivo}: Documentar todas as fases do projeto, incluindo a implementação, os resultados obtidos e as conclusões finais. Preparar a dissertação para submissão e defesa.
    
    \textbf{Tarefas}:
    \begin{itemize}
        \item Redação detalhada da dissertação.
        \item Revisão do documento com os orientadores.
        \item Preparação da apresentação final para a defesa.
    \end{itemize}
\end{enumerate}

