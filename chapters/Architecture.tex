\chapter{Prova de Consentimento}
\label{cap:arquitectura}

O presente capítulo descreve a arquitectura conceptual da solução proposta para a gestão e prova de consentimento digital. Pretende-se apresentar a visão geral do sistema, destacando os principais componentes, a lógica de funcionamento e os mecanismos criptográficos que garantem propriedades fundamentais como autenticidade, integridade, não repúdio, transparência e auditabilidade. Através desta abordagem, é possível assegurar que cada decisão do utilizador é registada de forma verificável e imutável.

\section{Visão Geral da Arquitetura}

É proposto um sistema que define um fluxo de consentimento no qual cada decisão do utilizador é registada, assinada e validada juntamente com o \textit{Provedor de Serviços}, assegurando que nenhuma das partes pode manipular ou negar a informação posteriormente. Para tal, existem duas entidades principais:

\begin{itemize}
    \item \acrfull{ds}: o utilizador final que interage com a interface web do serviço e fornece o consentimento no qual tem uma extensão de navegador à escuta dos eventos do navegador. O \acrshort{ds} é responsável por assinar digitalmente o consentimento antes de o enviar para validação, criando uma prova verificável da sua decisão.
    \item \acrfull{sp}: o prestador de serviços que disponibiliza o website com o \acrshort{cmp} à escolha e mantém um servidor para a troca de informação. Isto é, recebe os consentimentos assinados pelo \acrshort{ds}, valida a assinatura do utilizador, assina novamente o consentimento e mantém um registo imutável. Este registo permite auditoria, revogação e consulta futura.
\end{itemize}

O fluxo completo do sistema pode ser descrito de forma conceptual:

\begin{enumerate}
    \item O \acrshort{ds} interage com o banner de consentimento apresentado pelo \acrshort{cmp} escolhido na interface web fornecida pelo \acrshort{sp}.
    \item A extensão do navegador do \acrshort{ds} captura o evento, prepara o consentimento e assina digitalmente os dados.
    \item O consentimento assinado é enviado ao servidor do \acrshort{sp}, que valida a assinatura do \acrshort{ds} e cria um registo final, incorporando a assinatura do \acrshort{sp}.
    \item O registo resultante é devolvido ao \acrshort{ds}, que pode validar a assinatura do \acrshort{sp}, garantindo que o consentimento foi corretamente registado e não foi alterado.
    \item Ambos, \acrshort{ds} e \acrshort{sp}, mantêm cópias do registo, criando um histórico verificável e auditável. Posteriormente, este pode ser enviado para uma \textit{third-party}
\end{enumerate}

Este processo pode ser observado de forma mais clara na Figura~\ref{fig:swimlane1}, que ilustra o fluxo completo de interações entre as diferentes entidades do sistema.

Desta forma, o sistema garante quatro propriedades fundamentais. Em primeiro lugar, assegura a \textit{transparência}, uma vez que todos os passos do processo podem ser verificados tanto pelo utilizador como pelo prestador de serviços. Em segundo lugar, garante a \textit{autenticidade} e \textit{integridade} dos consentimentos, através do uso de assinaturas digitais que impedem alterações e confirmam a proveniência das entidades envolvidas. A terceira propriedade é o \textit{não repúdio}, que impede o \acrshort{ds} de negar a sua decisão e o \acrshort{sp} de alegar que não recebeu ou validou o consentimento. Por fim, o sistema promove a \textit{auditabilidade}, mantendo um histórico de consentimentos acessível a ambas as partes, o que permite cumprir os requisitos legais e regulatórios em matéria de proteção de dados.

\section{Componentes}

O sistema é constituído por três componentes principais: a \textit{interface web}, a \textit{extensão do navegador} e o \textit{servidor}.
A interface web representa o ponto de contacto direto com o utilizador e disponibiliza o \textit{banner} de consentimento.
A extensão do navegador actua como intermediário, recebendo os eventos provenientes da interface web feitos pelo utilizador e trata, no background, do processo de assinatura digital como também a autenticação do utilizador através do envio do certificado digital do mesmo.
Por fim, existe um servidor com o qual são trocados os certificados e assinaturas, permitindo que, no final do processo, seja obtido um registo estruturado de consentimento auditável e verificável. Este é responsável por validar as assinaturas recebidas e criar um registo imutável com o consentimento assinado por ambas as partes.

\section{Modelo de confiança}

O processo de recolha e gestão de consentimentos digitais enfrenta vários desafios de confiança.
Em particular, surgem questões relacionadas com a falta de garantias sobre a identidade das entidades envolvidas e a ausência de mecanismos claros de auditoria. 
Estes problemas levantam dúvidas tanto para os \acrshort{ds}, que necessitam de garantias de transparência e controlo sobre as suas decisões, como para os \acrshort{sp}, que precisam de provas fiáveis para demonstrar conformidade regulatória.

Para colmatar estas lacunas, o modelo de confiança proposto assenta em duas camadas principais de proteção: 
\textit{certificados digitais}, que funcionam como método de autenticação das entidades envolvidas, e \textit{assinaturas digitais}, que asseguram a autenticidade, a integridade e o não repúdio das decisões de consentimento. 
Graças a estes mecanismos, cada interação é não só verificável por ambas as partes, como também \textit{auditável}, permitindo a reconstrução fiel do histórico de consentimentos e reforçando a conformidade com requisitos legais como o \acrshort{rgpd}.

\subsection{Certificados digitais}

Os certificados desempenham um papel fundamental na garantia de identidade e na criação de comunicações seguras. De forma simplificada, um certificado digital é um ficheiro que associa uma chave pública a uma entidade (por exemplo, um utilizador ou um servidor). Esta associação é validada por uma \acrlong{ca} (\acrshort{ca}), que funciona como uma entidade de confiança responsável por emitir e assinar certificados.

A chave privada deve permanecer confidencial, pois é utilizada para operações críticas, como a criação de assinaturas digitais. Já a chave pública, incluída no certificado, pode ser partilhada e serve para validar essas assinaturas.

A \acrshort{ca} raiz (root \acrshort{ca}) é a autoridade de topo, responsável por assinar certificados de entidades intermédias ou diretamente de clientes e servidores, porém esta última trata-se de uma má prática, normas internacionais como a ETSI EN 319 411 \citep{ETSI_EN_319_411_1} proíbem explicitamente o uso direto da Root CA para emitir certificados operacionais. Este mecanismo hierárquico garante que, ao receber um certificado, é possível verificar a sua autenticidade através da cadeia de confiança estabelecida pela autoridade certificadora.

\subsection{Assinatura Digital}

A assinatura digital é um mecanismo criptográfico baseado em criptografia de chave pública, concebido para garantir a autenticidade e a integridade de uma mensagem ou documento eletrónico. O seu funcionamento baseia-se em dois elementos fundamentais: a chave privada e a chave pública. A chave privada é utilizada para gerar a assinatura digital e deve permanecer secreta, acessível apenas ao titular. A chave pública é distribuída, neste caso através de certificados digitais, permitindo que qualquer entidade verifique a validade da assinatura.

Desta forma, a assinatura digital assegura três propriedades essenciais \citep{digitalsignatures}:
\begin{itemize}
    \item \textbf{Autenticidade}: confirma que a mensagem foi assinada pela entidade detentora da chave privada correspondente.
    \item \textbf{Integridade}: garante que o conteúdo não foi alterado após a assinatura.
    \item \textbf{Não repúdio}: impede que o autor negue a sua participação no processo, tornando a assinatura uma evidência legalmente relevante.
\end{itemize}

As assinaturas digitais constituem o fundamento de sistemas confiáveis de registo de consentimentos, transações financeiras e documentos eletrónicos, sendo amplamente utilizadas em padrões de segurança e infraestruturas de chave pública.

\section{Registo do consentimento}

O consentimento do utilizador é preferencialmente um registo estruturado que pode ser interpretado e processado de forma padronizada. Este registo deve ser interoperável, auditável e verificável, permitindo que diferentes sistemas o leiam e validem sem ambiguidade.

Para garantir estas propriedades, o consentimento é assinado digitalmente tanto pelo utilizador como pela entidade que o recebe. A troca de certificados entre as partes possibilita a verificação mútua das assinaturas, reforçando a confiança no processo. O objecto resultante agrega informação relevante sobre as decisões do utilizador, bem como metadados necessários à validação, mantendo a integridade e autenticidade do consentimento.

Adotar um padrão estruturado para o consentimento permite:
\begin{itemize}
    \item Facilitar a integração com diferentes sistemas/aplicações;
    \item Manter um registo auditável e verificável ao longo do tempo;
\end{itemize}

A ausência de cifragem no registo de consentimento simplifica o processo de verificação e auditoria, mas aumenta o risco de exposição de informação pessoal. Mesmo que o registo não contenha dados diretamente identificáveis, como o nome ou o e-mail, pode incluir identificadores indiretos, por exemplo, identificadores únicos, que podem permitir reconhecer ou reidentificar o utilizador. Para reduzir este risco e garantir conformidade com o princípio da confidencialidade previsto nas legislações, recomenda-se a aplicação de técnicas complementares de pseudonimização ou cifragem seletiva, de modo a equilibrar a transparência com a proteção da privacidade do utilizador.

\section{Interacção Entre Componentes}

A figura~\ref{fig:swimlane1} ilustra o fluxo completo de troca de mensagens e assinaturas durante o processo de consentimento. O sistema envolve três componentes principais: o \textit{cliente}, que inclui a extensão do navegador para capturar e assinar digitalmente os consentimentos; a \textit{interface web} do \textit{prestador de serviços}, que apresenta o \textit{banner} de consentimento; e o \textit{servidor}, responsável por validar e assinar os registos de consentimento.

Quando o utilizador acede ao website, a interface web apresenta o \textit{banner} de consentimento e juntamente envia o certificado do servidor. O utilizador preenche o banner e, através da extensão do navegador, assina digitalmente o consentimento utilizando a sua chave privada. A extensão prepara então o objeto final do consentimento, que inclui a assinatura do utilizador, e envia-o para o servidor juntamente com o certificado do cliente.

O servidor valida a assinatura do utilizador e o respetivo certificado, gera a sua própria assinatura digital sobre o consentimento e cria um registo final que agrega ambas as assinaturas. Este registo é devolvido ao cliente, que valida a assinatura do servidor e mantém uma cópia do consentimento de forma verificável.

Desta forma, o fluxo garante que, tanto o cliente como o servidor, dispõem de provas verificáveis e mutuamente validadas do consentimento, assegurando transparência, integridade e auditabilidade ao longo de todo o processo.

\label{fig:swimlane1}
\begin{figure}[h]
\begin{center}
\includegraphics[width=1\textwidth]{images/swimlanes.png}
\end{center}
\caption{Diagrama do protocolo de prova de consentimento}
\end{figure}

\newpage

\section{Lógica de Funcionamento}

A estratégia proposta assenta numa abordagem de assinaturas digitais, na qual o cliente e servidor participam ativamente na criação de um registo de consentimento verificável e imutável. Esta abordagem garante quatro propriedades fundamentais: \textit{transparência}, na medida em que todos os passos podem ser verificados; \textit{descentralização}, dado que nenhuma das partes detém controlo unilateral; \textit{não-repúdio}, assegurando que os consentimentos prestados não podem ser posteriormente negados; e \textit{auditabilidade}, uma vez que o histórico completo permanece acessível tanto localmente, junto de cada entidade participante, como opcionalmente através de um serviço de terceiros responsável pela verificação ou conservação dos registos.

\section{Benefícios}

A \textit{Prova de Consentimento} proposta apresenta benefícios distintos para os diferentes intervenientes.
%Do ponto de vista do utilizador, a solução facilita aceder aos consentimentos prestados mais facilmente, assegurando transparência no processo.
Do ponto de vista do utilizador, este tem a possibilidade de verificar a integridade dos registos, garantindo que não foram alvo de manipulação, e dispõe ainda de mecanismos que lhe conferem autonomia para comprovar eventuais incumprimentos por parte do prestador de serviços.

Para as organizações, a arquitectura disponibiliza provas fiáveis de consentimentos válidos, que podem ser utilizadas em auditorias ou processos de verificação de conformidade.
Deste modo, contribui para a redução dos riscos associados ao incumprimento das regulamentações em vigor, promovendo maior confiança e segurança jurídica no tratamento de dados pessoais.

\section{Síntese do capítulo}

Neste capítulo foi apresentada a arquitectura conceptual da solução de prova de consentimento, detalhando os principais componentes, a lógica de funcionamento e os mecanismos criptográficos, como certificados digitais e assinaturas digitais. Através da Prova de Conceito, demonstrou-se a viabilidade do modelo conceptual, validando a autenticidade, integridade, não repúdio, transparência e auditabilidade dos registos de consentimento num ambiente simplificado, sem recorrer ainda a uma implementação completa.

O capítulo seguinte descreve a implementação prática desta arquitectura, materializada num protótipo funcional que integra os princípios validados na Prova de Conceito. Nesta fase, são exploradas tecnologias reais e fluxos completos, permitindo avaliar o desempenho, a interoperabilidade e a experiência de utilização, evidenciando os benefícios e limitações do sistema face às limitações identificadas no trabalho relacionado (\ref{cap:relacionado}).
