\chapter{Introdução}
\label{cap:introducao}

Com a crescente utilização de plataformas online, a quantidade de dados pessoais recolhidos de utilizadores tem aumentado. Contudo, muitos utilizadores desconhecem como esses dados são recolhidos, processados e partilhados pelas entidades envolvidas com o mesmo processamento. Quando um utilizador acede a um site, frequentemente, não tem uma visão clara sobre o que acontece com os dados fornecidos, nem se o servidor recebeu exatamente aquilo a que deu consentimento. Este cenário levanta questões de privacidade, especialmente, no que diz respeito ao cumprimento de regulamentações como o \acrfull{rgpd} (respetiva à União Europeia) e outras legislações de privacidade.

As plataformas de Gestão de Consentimento (\acrshort{cmp}s) surgem como soluções para o processamento de dados, uma vez que asseguram aos serviços das empresas a possibilidade de poderem cumprir legislações de tratamento de dados, tais como, o \acrshort{rgpd} \citep{Santos2021}, permitindo a recolha e gestão do consentimento dos utilizadores para a utilização dos seus dados. Contudo, as \acrshort{cmp}s tradicionais, muitas das quais são soluções fechadas, apresentam limitações em termos de transparência e auditoria. 
Estas plataformas, no entanto, não conseguem fornecer uma visão clara sobre o que realmente acontece com os dados depois de o utilizador ter feito as escolhas em relação ao seu consentimento, nem se a informação transmitida ao servidor está em conformidade com essas mesmas decisões. A falta de um método de auditoria transparente torna difícil o utilizador provar a não conformidade das empresas que não estão a respeitar o consentimento e a cumprir as regulamentações de privacidade.

\section{Motivação}

Uma das falhas dos modelos atuais reside no risco de que as escolhas de privacidade e as configurações de consentimento do utilizador não sejam efetivamente respeitadas no fluxo de dados. Num cenário de litígio ou disputa de privacidade, as ferramentas tradicionais não oferecem uma forma de prova criptográfica, imutável e verificável.
A motivação para este projeto surge então precisamente da falta da auditoria desse fluxo de dados, oferecendo um mecanismo de prova, que permite demonstrar o incumprimento das preferências do utilizador, caso tal seja identificado.
Uma das grandes limitações das ferramentas tradicionais é o facto de muitas delas serem soluções fechadas que não transmitem a maior transparência possível, dificultando a compreensão e auditoria do que acontece com os dados.
Este projeto visa colmatar a falta de transparência, fornecendo uma solução que não dependa da confiança cega nos registos internos do responsável pelo tratamento. Este registo é robusto o suficiente para servir como evidência inquestionável caso seja detetado um incumprimento, permitindo que o utilizador verifique de forma independente o histórico completo e não adulterado do processamento dos seus dados.
% nao discutir escolhas feifas no trabalho, apenas motivacao
% Por isso, optou-se por explorar soluções de tratamento de dados, que, ao serem mais acessíveis, permitem uma maior transparência e auditabilidade dos dados recolhidos. Através dessas plataformas, é possível compreender melhor os fluxos de dados e garantir que o processo de consentimento seja claro e auditável.

\section{Objetivos}

Criar uma solução que permite, não só recolher e gerir o consentimento dos utilizadores, mas também permitir comprovar em caso de incumprimento dos consentimento os dados de forma transparente.
Explorar as tecnologias de tratamentos de dados e permitir registar de forma imutável as interações e escolhas dos utilizadores, de modo a assegurar a conformidade e aumentar a confiança no processo. 
Desenhar e prototipar um sistema que ajuda na gestão dos consentimentos de um utilizador, uma vez que se integra a recolha de consentimento com mecanismos de prova, utilizando estratégias criptográficas, de forma a criar registos verificáveis, imutáveis e não repudiáveis.
O objetivo não é garantir que os \acrfull{sp} cumprirão integralmente regulamentos como o \acrshort{rgpd}, pois a conformidade final depende das suas práticas internas. Em vez disso, a solução visa promover a transparência e, fundamentalmente, fornecer a evidência técnica irrefutável que prove se as decisões dos utilizadores foram (ou não foram) respeitadas no tratamento dos dados armazenados.
Desta forma, o sistema capacita os utilizadores a provar a não-conformidade em caso de violação das escolhas de privacidade.

% Isto não são objetivos mas sim metodologia. Parece demasiado relatório
% Para atingir este objetivo, será realizada inicialmente uma revisão de literatura sobre \acrshort{cmp}s, \acrshort{rgpd} e a utilização de \textit{blockchain} para auditoria de consentimento e integração de contratos inteligentes. Em seguida, proceder-se-á à análise das abordagens existentes para plataformas de gestão de consentimento, identificando as suas limitações e as melhores alternativas para o nosso sistema. Com base nesta análise, foi desenhado um modelo arquitetural para uma plataforma que integre auditoria transparente e registo imutável de consentimentos. Posteriormente, foi desenvolvido uma prova de conceita utilizando essa mesma arquitetura, com a integração um \acrshort{cmp} Open-source. O sistema foi testado e otimizado, avaliando-se o seu desempenho, segurança e usabilidade.

\section{Contribuições}

As contribuições desta dissertação podem ser agrupadas em dois níveis complementares: conceptual e prático. 
No plano conceptual, destaca-se o desenho de uma arquitetura genérica para a gestão de consentimento. Esta arquitetura foi concebida de forma modular e independente de tecnologias específicas, permitindo que qualquer pessoa ou organização a possa implementar segundo os seus próprios requisitos. Para reforçar a generalidade da proposta, optou-se por separar a descrição da arquitetura da implementação concreta, salientando a sua reusabilidade e extensibilidade.

No plano prático, apresenta-se a proposta e a implementação de uma solução funcional que materializa a arquitetura definida. O protótipo desenvolvido baseia-se em tecnologias \textit{open-source} e integra mecanismos de assinatura digital, certificados e \textit{JSON Web Signatures} (JWS), demonstrando a viabilidade da abordagem proposta. Entre os contributos técnicos concretos incluem-se a criação de uma extensão de navegador (\textit{Firefox}), o desenvolvimento de um servidor, a integração com o \acrshort{cmp} \textit{open-source} Klaro.js e a gestão de certificados digitais, garantindo que o processo de consentimento é registado de forma transparente e auditável.

Em conjunto, estas contribuições oferecem não só uma abordagem para a recolha e comprovação de consentimentos, mas também uma solução prática que pode ser replicada e adaptada em diferentes contextos.

\section{Estrutura da dissertação}

A presente dissertação encontra-se organizada em três partes principais, compreendendo um total de seis capítulos.

Na primeira parte é introduzido o problema (Capítulo~\ref{cap:introducao}), bem como a motivação, os objectivos e o estudo do trabalho relacionado, apresentando algumas das soluções existentes para problemas semelhantes (Capítulo~\ref{cap:relacionado}).

A segunda parte descreve os contributos principais do trabalho, incluindo o desenho detalhado da arquitectura proposta (Capítulo~\ref{cap:arquitectura}), a implementação da solução e a respectiva análise (Capítulo~\ref{cap:implementacao}), e as conclusões gerais acompanhadas de perspectivas de trabalho futuro (Capítulo~\ref{cap:conclusoes}).

A terceira parte é dedicada à avaliação, onde se discute o funcionamento e a validação da solução proposta (Capítulo~\ref{cap:avaliacao}).

Por fim, os apêndices apresentam material de apoio, detalhes complementares dos resultados experimentais, listagens de código relevante e documentação das ferramentas utilizadas.
