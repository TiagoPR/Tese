\chapter{Introdução}

%Contexto, motivação, principais objetivos.

\section{Contextualização}

Com a digitalização em crescimento e a utilização crescente de plataformas online, a quantidade de dados pessoais recolhidos dos utilizadores tem aumentado exponencialmente. Contudo, muitos utilizadores desconhecem como esses dados são recolhidos, processados e partilhados pelos servidores. Quando um utilizador acede a um site, ele frequentemente não tem uma visão clara sobre o que acontece com os dados fornecidos nem se o servidor recebeu exatamente aquilo a que deu consentimento. Este cenário levanta questões de privacidade, especialmente no que diz respeito ao cumprimento de regulamentações como o \acrfull{rgpd} e outras legislações de privacidade.

No contexto atual, as Plataformas de Gestão de Consentimento (\acrshort{cmp}s) surgem como ferramentas que dizem assegurar às plataformas das empresas que assim podem cumprir o \acrshort{rgpd} \citep{Santos2021}, permitindo a recolha e gestão do consentimento dos utilizadores para a utilização dos seus dados. Contudo, as \acrshort{cmp}s tradicionais, muitas das quais são soluções fechadas, apresentam limitações em termos de transparência e auditoria. Estas plataformas não fornecem uma visão clara sobre o que realmente acontece com os dados depois de o utilizador ter dado o seu consentimento, nem se a informação transmitida ao servidor está em conformidade com a escolha do utilizador. A falta de uma auditoria transparente torna difícil garantir que as empresas estão a respeitar o consentimento dos utilizadores e a cumprir as regulamentações de privacidade.

\section{Motivação}

A motivação para este projeto surge precisamente da necessidade de permitir a auditoria desse fluxo de dados, oferecendo uma forma de garantir que as escolhas do utilizador são efetivamente respeitadas. Uma das grandes limitações das \acrshort{cmp}s tradicionais é o facto de muitas delas serem soluções fechadas, dificultando a compreensão e auditoria do que acontece com os dados. Por isso, optou-se por explorar soluções de \acrshort{cmp} \textit{open source}, que, ao serem mais acessíveis, permitem uma maior transparência e auditabilidade dos dados recolhidos. Através dessas plataformas, é possível compreender melhor os fluxos de dados e garantir que o processo de consentimento seja claro e auditável.

\section{Objetivos}

O objetivo deste trabalho é criar uma solução que permita não só recolher e gerir o consentimento dos utilizadores, mas também auditar o processo de forma transparente e verificável. Uma possível solução para isso é o uso de tecnologias como \textit{blockchain}, que, ao permitir registar de forma imutável as interações e escolhas dos utilizadores, pode assegurar a conformidade e aumentar a confiança no processo. O principal objetivo deste trabalho é, portanto, desenhar e prototipar um sistema que ajuda na gestão do consentimento do utilizador uma vez que se integra a recolha de consentimento com mecanismos de auditoria transparentes, utilizando tecnologias como \textit{blockchain}, de forma a criar transparência entre as decisões dos utilizadores e os dados que são armazenados nas \acrshort{cmp} e que esteja conforme o \acrshort{rgpd}.

Para atingir este objetivo, será realizada inicialmente uma revisão de literatura sobre \acrshort{cmp}s, \acrshort{rgpd} e a utilização de \textit{blockchain} para auditoria de consentimento e integração de contratos inteligentes. Em seguida, proceder-se-á à análise das abordagens existentes para plataformas de gestão de consentimento, identificando as suas limitações e as melhores alternativas para o nosso sistema. Com base nesta análise, será desenhado um modelo arquitetural para uma plataforma que integre auditoria transparente e registo imutável de consentimentos atráves de uma rede de \textit{blockchain} (ex. Ethereum). Posteriormente, será desenvolvido um protótipo funcional da solução proposta, garantindo a integração entre a \acrshort{cmp} e um registo \textit{blockchain}. O sistema será testado e otimizado, avaliando-se o seu desempenho, segurança e usabilidade.

%\section{Estrutura da dissertação}


