\chapter{Conclusões e trabalho futuro}
\label{cap:conclusoes}

\section{Conclusões}

O principal objetivo definido para este trabalho consistia em conceber e prototipar um sistema de gestão de consentimento que, para além de recolher as escolhas do utilizador, permitisse também auditar o processo de forma transparente e verificável caso fosse necessário. Para tal, partiu-se da análise crítica das limitações das \acrshort{cmp}s tradicionais, em particular a falta de transparência e de mecanismos de auditoria, e delineou-se uma solução que permitisse acrescentar estas garantias independentemente da plataforma utilizada. A proposta sugere funcionalidades que podem ser aplicados em qualquer \acrshort{cmp} existente, potenciando a transparência e a verificabilidade do processo de gestão de consentimento.

Este objetivo foi alcançado através da definição de uma arquitetura genérica baseada em três entidades principais (utilizador, extensão no navegador e servidor), capaz de registar consentimentos com recurso a assinaturas digitais e certificados. A implementação prática validou a viabilidade deste modelo, recorrendo a tecnologias como Node.js no servidor, Klaro.js para o \textit{banner} de consentimento e uma extensão de navegador em JavaScript, suportada por bibliotecas de criptografia. O processo resultou num fluxo completo de recolha, assinatura, validação e registo de consentimentos em formato JWS, garantindo autenticidade, integridade, não repúdio e auditabilidade.

Desta forma, pode afirmar-se que os objetivos delineados foram cumpridos. A solução concebida não só demonstra ser possível conjugar simplicidade de utilização com garantias de segurança e confiança, como também responde diretamente ao principal problema identificado: a ausência de mecanismos que permitam ao utilizador salvaguardar-se em caso de não conformidade. Com o registo duplamente assinado em formato JWS, o utilizador dispõe de uma prova verificável do consentimento que efetivamente forneceu, podendo assim contestar eventuais falhas do lado do prestador de serviços. Este mecanismo de auditoria constitui o contributo mais relevante desta dissertação, ao assegurar que tanto utilizador como organização partilham um histórico comum, transparente e verificável.

%Apesar de não ter sido integrada nesta fase, a extensão do workflow com blockchain foi também considerada, tendo sido apresentado um esboço de como poderia reforçar a imutabilidade e a confiança nos registos de consentimento. Este ponto abre caminho para trabalho futuro, assim como a avaliação em cenários mais próximos de ambientes de produção, permitindo aferir o desempenho, escalabilidade e aplicabilidade em contextos organizacionais de maior dimensão.

\section{Trabalho futuro}

Embora a solução apresentada tenha demonstrado a viabilidade de integrar mecanismos de auditoria e verificação de consentimentos em plataformas existentes, permanecem diversas oportunidades para evolução e aprofundamento do trabalho. Esta secção discute possíveis direções para trabalhos futuros, destacando melhorias técnicas e expansão de funcionalidades que possam aumentar a transparência, a confiabilidade e a escalabilidade do sistema. O objetivo é fornecer uma perspetiva sobre como a investigação presente pode ser prolongada, contribuindo para soluções mais robustas e abrangentes no domínio da gestão de consentimento de dados.

\subsection{Complementação do fluxo com \textit{blockchain} e contratos inteligentes}

A prova de conceito desenvolvida assegura a autenticidade, integridade e verificabilidade dos consentimentos através de assinaturas digitais e certificados, no entanto, uma possível extensão futura seria a integração com um \textit{ledger} público.

A utilização desta tecnologia permitiria reforçar a imutabilidade dos registos, uma vez que o consentimento poderia ser armazenado de forma distribuída numa rede pública, resistente a manipulações. Neste cenário, após o consentimento ser validado e assinado por ambas as partes, o servidor poderia proceder ao registo do identificador único associado ao consentimento (ID) numa \textit{blockchain} pública, como a \textit{Ethereum}.

Com este mecanismo, seria possível garantir que qualquer entidade interessada, incluindo o próprio utilizador, pudesse verificar, de forma independente, a existência e consistência dos consentimentos registados. O registo em \textit{blockchain} funcionaria, assim, como uma prova adicional de confiança, complementando as garantias já oferecidas pelo sistema atual.  

Importa salientar que, na integração com \textit{blockchain}, apenas identificadores ou \textit{hashes} dos consentimentos seriam registados, enquanto os conteúdos permaneceriam cifrados. Caso os dados completos fossem colocados diretamente na \textit{blockchain} sem cuidado, existiria o risco de rastreabilidade das ações dos utilizadores, permitindo a monitorização das suas escolhas de consentimento e atividade online. Ao manter os conteúdos cifrados, esta abordagem assegura que a imutabilidade e auditabilidade do registo não comprometem a privacidade nem a confidencialidade dos utilizadores.

Além disso, esta abordagem abriria caminho para auditorias independentes e mecanismos automatizados de verificação, assegurando que os consentimentos mantêm-se inalterados ao longo do tempo e promovendo uma maior transparência na gestão de dados pessoais.

A exploração desta integração com \textit{blockchain} e contratos inteligentes constitui, portanto, um rumo relevante para trabalho futuro, potenciando a robustez e a confiança da solução aqui apresentada.

\subsection{Integração com keychain do sistema do cliente}

Outra direção futura consiste em explorar a integração da extensão com a importação da \textit{keychain} do sistema operativo do utilizador. Esta integração permitiria gerir de forma mais segura as chaves privadas utilizadas na assinatura dos consentimentos, reduzindo o risco de exposição ou uso indevido. Além disso, garantiria maior comodidade para o utilizador, que não precisaria de gerir manualmente chaves criptográficas nem depender de soluções externas para armazenar credenciais sensíveis.

\subsection{Disponibilização da solução como projeto Open-source}

Um objetivo adicional é tornar a solução disponível como projeto Open-source. Esta abordagem facilitaria auditorias independentes, fomentaria a confiança por parte dos utilizadores e permitiria que a comunidade contribuísse para melhorias, correções de segurança e evolução da plataforma. A abertura do código reforçaria ainda a transparência e a auditabilidade do sistema, promovendo uma adoção mais ampla em contextos académicos, corporativos e de investigação.
